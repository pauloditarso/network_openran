% Parte 4: Novos Paradigmas e Futuro
\section{Novos Paradigmas e Futuro}

\begin{frame}{Novos Paradigmas e Futuro}
\small
\begin{itemize}
  \item \textbf{Redes privadas 5G/6G}: possibilitam aplicações industriais, smart cities e setores críticos (energia, saúde, logística), com maior controle sobre segurança, QoS e confiabilidade.
  \item \textbf{IA/ML para otimização da RAN}: uso de algoritmos inteligentes para ajuste dinâmico de parâmetros, predição de tráfego, alocação eficiente de recursos e manutenção preditiva.
  \item \textbf{Orquestração com NFV, SDN e Edge}: integração de funções virtualizadas, controle centralizado e computação na borda para oferecer baixa latência e maior flexibilidade no provisionamento de serviços.
  \item \textbf{Caminho para o 6G}: evolução em direção a redes autônomas, com autoconfiguração e auto-recuperação, além de suportar casos de uso avançados como comunicação holográfica, internet tátil e integração massiva de dispositivos inteligentes.
\end{itemize}
\end{frame}
